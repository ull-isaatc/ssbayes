\pagestyle{scrheadings}
\ihead[]{\rightmark}
\ohead[]{Iván Rodríguez Méndez}
\ofoot[]{\thepage{}}
\chapter{Toolbox utilizada en Matlab.}\label{ApendiceB}
Para la realización de este trabajo hemos utilizado una toolbox de MATLAB realizada por Jouni Hartikainen, Arno Solin y Simo Särkkä.
El objetivo de esta toolbox es implementar todas las funciones vistas en el capítulo \ref{ch:capitulo3} de este trabajo y muchas otras.
Para una mayor comprensión sobre la propia toolbox y los códigos que se presentan en algunos capítulos de este trabajo introduciremos las funciones que hemos utilizado (únicamente las básicas sobre los filtros), para ello introduciremos los segmentos de código de MATLAB con los parámetros de cada función explicados.
Todas aquellas que funciones que son llamadas dentro de la implementación de los filtros no las incluiremos en este apéndice ya que haría su lectura demasiado engorrosa.
Para encontrar más información acerca del resto de funciones puede consultarse \cite{toolbox_simo}, donde encontrarán la toolbox para su descarga además de su documentación.

\section{Filtro clásico de Kalman}
\subsection{Ciclo de predicción}
\lstset{language=Matlab, breaklines=true, basicstyle=\footnotesize}
\lstset{numbers=left, numberstyle=\tiny, stepnumber=1, numbersep=-2pt}
\begin{lstlisting}[frame=single]
 %KF_PREDICT  Perform Kalman Filter prediction step
% Syntax:
%   [X,P] = KF_PREDICT(X,P,A,Q,B,U)
% In:
%   X - Nx1 mean state estimate of previous step
%   P - NxN state covariance of previous step
%   A - Transition matrix of discrete model (optional, default identity)
%   Q - Process noise of discrete model (optional, default zero)
%   B - Input effect matrix (optional, default identity)
%   U - Constant input (optional, default empty)
%
% Out:
%   X - Predicted state mean
%   P - Predicted state covariance
% Description:
%   Perform Kalman Filter prediction step. The model is
%     x[k] = A*x[k-1] + B*u[k-1] + q,  q ~ N(0,Q).
%   The predicted state is distributed as follows:
%     p(x[k] | x[k-1]) = N(x[k] | A*x[k-1] + B*u[k-1], Q[k-1])
%   The predicted mean x-[k] and covariance P-[k] are calculated
%   with the following equations:
%     m-[k] = A*x[k-1] + B*u[k-1]
%     P-[k] = A*P[k-1]*A' + Q.
%   If there is no input u present then the first equation reduces to
%     m-[k] = A*x[k-1]
function [x,P] = kf_predict(x,P,A,Q,B,u)
  % Check arguments
  %
  if nargin < 3
    A = [];
  end
  if nargin < 4
    Q = [];
  end
  if nargin < 5
    B = [];
  end
  if nargin < 6
    u = [];
  end  
  %
  % Apply defaults
  %
  if isempty(A)
    A = eye(size(x,1));
  end
  if isempty(Q)
    Q = zeros(size(x,1));
  end
  if isempty(B) & ~isempty(u)
    B = eye(size(x,1),size(u,1));
  end
  %
  % Perform prediction
  %
  if isempty(u)
    x = A * x;
    P = A * P * A' + Q;
  else
    x = A * x + B * u;
    P = A * P * A' + Q;
  end

\end{lstlisting}

\subsection{Ciclo de actualización}
\lstset{language=Matlab, breaklines=true, basicstyle=\footnotesize}
\lstset{numbers=left, numberstyle=\tiny, stepnumber=1, numbersep=-2pt}
\begin{lstlisting}[frame=single]
%KF_UPDATE  Kalman Filter update step
% Syntax:
%   [X,P,K,IM,IS,LH] = KF_UPDATE(X,P,Y,H,R)
% In:
%   X - Nx1 mean state estimate after prediction step
%   P - NxN state covariance after prediction step
%   Y - Dx1 measurement vector.
%   H - Measurement matrix.
%   R - Measurement noise covariance.
% Out:
%   X  - Updated state mean
%   P  - Updated state covariance
%   K  - Computed Kalman gain
%   IM - Mean of predictive distribution of Y
%   IS - Covariance or predictive mean of Y
%   LH - Predictive probability (likelihood) of measurement. 
% Description:
%   Kalman filter measurement update step. Kalman Filter
%   model is
%     x[k] = A*x[k-1] + B*u[k-1] + q,  q ~ N(0,Q)
%     y[k] = H*x[k] + r,r ~ N(0,R)
%   Prediction step of Kalman filter computes predicted
%   mean m-[k] and covariance P-[k] of state:
%     p(x[k] | y[1:k-1]) = N(x[k] | m-[k], P-[k])
function [X,P,K,IM,IS,LH] = kf_update(X,P,y,H,R)
  % Check which arguments are there
  %
  if nargin < 5
    error('Too few arguments');
  end
  %
  % update step
  %
  IM = H*X;
  IS = (R + H*P*H');
  K = P*H'/IS;
  X = X + K * (y-IM);
  P = P - K*IS*K';
  if nargout > 5
    LH = gauss_pdf(y,IM,IS);
  end
\end{lstlisting}
\section{Filtro extendido de Kalman de primer orden}
\subsection{Ciclo de predicción}
\lstset{language=Matlab, breaklines=true, basicstyle=\footnotesize}
\lstset{numbers=left, numberstyle=\tiny, stepnumber=1, numbersep=-2pt}
\begin{lstlisting}[frame=single]
 %EKF_PREDICT1  1st order Extended Kalman Filter prediction step
% Syntax:
%   [M,P] = EKF_PREDICT1(M,P,[A,Q,a,W,param])
% In:
%   M - Nx1 mean state estimate of previous step
%   P - NxN state covariance of previous step
%   A - Derivative of a() with respect to state as
%       matrix, inline function, function handle or
%       name of function in form A(x,param)(optional, default eye())
%   Q - Process noise of discrete model(optional, default zero)
%   a - Mean prediction E[a(x[k-1],q=0)] as vector,
%       inline function, function handle or name
%       of function in form a(x,param)(optional, default A(x)*X)
%   W - Derivative of a() with respect to noise q
%       as matrix, inline function, function handle
%       or name of function in form W(x,param)(optional, default identity)
%   param - Parameters of a (optional, default empty)
% Out:
%   M - Updated state mean
%   P - Updated state covariance
% Description:
%   Perform Extended Kalman Filter prediction step.
function [M,P] = ekf_predict1(M,P,A,Q,a,W,param)
  % Check arguments
  %
  if nargin < 3
    A = [];
  end
  if nargin < 4
    Q = [];
  end
   if nargin < 5
    a = [];
  end
  if nargin < 6
    W = [];
  end
  if nargin < 7
    param = [];
  end
  % Apply defaults
  if isempty(A)
    A = eye(size(M,1));
  end
  if isempty(Q)
    Q = zeros(size(M,1));
  end
  if isempty(W)
    W = eye(size(M,1),size(Q,2));
  end
  if isnumeric(A)
    % nop
  elseif isstr(A) | strcmp(class(A),'function_handle')
    A = feval(A,M,param);
  else
    A = A(M,param);
  end
  % Perform prediction
  if isempty(a)
    M = A*M;
  elseif isnumeric(a)
    M = a;
  elseif isstr(a) | strcmp(class(a),'function_handle')
    M = feval(a,M,param);
  else
    M = a(M,param);
  end
  if isnumeric(W)
    % nop
  elseif isstr(W) | strcmp(class(W),'function_handle')
    W = feval(W,M,param);
  else
    W = W(M,param);
  end

  P = A * P * A' + W * Q * W';
\end{lstlisting}
\subsection{Ciclo de actualización}
\lstset{language=Matlab, breaklines=true, basicstyle=\footnotesize}
\lstset{numbers=left, numberstyle=\tiny, stepnumber=1, numbersep=-2pt}
\begin{lstlisting}[frame=single]
%EKF_UPDATE1  1st order Extended Kalman Filter update step
% Syntax:
%   [M,P,K,MU,S,LH] = EKF_UPDATE1(M,P,Y,H,R,[h,V,param])
% In:
%   M  - Nx1 mean state estimate after prediction step
%   P  - NxN state covariance after prediction step
%   Y  - Dx1 measurement vector.
%   H  - Derivative of h() with respect to state as matrix,
%        inline function, function handle or name
%        of function in form H(x,param)
%   R  - Measurement noise covariance.
%   h  - Mean prediction (innovation) as vector,
%        inline function, function handle or name
%        of function in form h(x,param).(optional, default H(x)*X)
%   V  - Derivative of h() with respect to noise as matrix,
%        inline function, function handle or name
%        of function in form V(x,param).(optional, default identity)
%   param - Parameters of h (optional, default empty)
% Out:
%   M  - Updated state mean
%   P  - Updated state covariance
%   K  - Computed Kalman gain
%   MU - Predictive mean of Y
%   S  - Predictive covariance of Y
%   LH - Predictive probability (likelihood) of measurement.  
% Description:
%   Extended Kalman Filter measurement update step.
%   EKF model is
%     y[k] = h(x[k],r),r ~ N(0,R)
function [M,P,K,MU,S,LH] = ekf_update1(M,P,y,H,R,h,V,param)
  % Check which arguments are there
  if nargin < 5
    error('Too few arguments');
  end
  if nargin < 6
    h = [];
  end
  if nargin < 7
    V = [];
  end
  if nargin < 8
    param = [];
  end
  % Apply defaults
  %
  if isempty(V)
    V = eye(size(R,1));
  end
  % Evaluate matrices
  %
  if isnumeric(H)
    % nop
  elseif isstr(H) | strcmp(class(H),'function_handle')
    H = feval(H,M,param);
  else
    H = H(M,param);
  end
  if isempty(h)
    MU = H*M;
  elseif isnumeric(h)
    MU = h;
  elseif isstr(h) | strcmp(class(h),'function_handle')
    MU = feval(h,M,param);
  else
    MU = h(M,param);
  end
  if isnumeric(V)
    % nop
  elseif isstr(V) | strcmp(class(V),'function_handle')
    V = feval(V,M,param);
  else
    V = V(M,param);
  end
  % update step
  %  
  S = (V*R*V' + H*P*H');
  K = P*H'/S;
  M = M + K * (y-MU);
  P = P - K*S*K';

  if nargout > 5
    LH = gauss_pdf(y,MU,S);
  end
\end{lstlisting}
\section{Filtro \textit{unscented} de Kalman \textit{NonAugmented}}
\subsection{Ciclo de predicción}
\lstset{language=Matlab, breaklines=true, basicstyle=\footnotesize}
\lstset{numbers=left, numberstyle=\tiny, stepnumber=1, numbersep=-2pt}
\begin{lstlisting}[frame=single]
%UKF_PREDICT1  Nonaugmented (Additive) UKF prediction step
% Syntax:
%   [M,P] = UKF_PREDICT1(M,P,f,Q,f_param,alpha,beta,kappa,mat)
% In:
%   M - Nx1 mean state estimate of previous step
%   P - NxN state covariance of previous step
%   f - Dynamic model function as a matrix A defining
%       linear function a(x) = A*x, inline function,
%       function handle or name of function in
%       form a(x,param)(optional, default eye())
%   Q - Process noise of discrete model(optional, default zero)
%   f_param - Parameters of f (optional, default empty)
%   alpha - Transformation parameter(optional)
%   beta  - Transformation parameter(optional)
%   kappa - Transformation parameter(optional)
%   mat   - If 1 uses matrix form (optional, default 0)
% Out:
%   M - Updated state mean
%   P - Updated state covariance
% Description:
%   Perform additive form Unscented Kalman Filter prediction step.
%   Function a should be such that it can be given
%   DxN matrix of N sigma Dx1 points and it returns 
%   the corresponding predictions for each sigma
%   point. 
function [M,P,D] = ukf_predict1(M,P,f,Q,f_param,alpha,beta,kappa,mat)
  % Check which arguments are there
  if nargin < 2
    error('Too few arguments');
  end
  if nargin < 3
    f = [];
  end
  if nargin < 4
    Q = [];
  end
  if nargin < 5
    f_param = [];
  end
  if nargin < 6
    alpha = [];
  end
  if nargin < 7
    beta = [];
  end
  if nargin < 8
    kappa = [];
  end
  if nargin < 9
    mat = [];
  end
  % Apply defaults
  %
  if isempty(f)
    f = eye(size(M,1));
  end
  if isempty(Q)
    Q = zeros(size(M,1));
  end
  if isempty(mat)
    mat = 0;
  end
  % Do transform
  % and add process noise
  tr_param = {alpha beta kappa mat};
  [M,P,D] = ut_transform(M,P,f,f_param,tr_param);
  P = P + Q;
\end{lstlisting}
\subsection{Ciclo de actualización}
\lstset{language=Matlab, breaklines=true, basicstyle=\footnotesize}
\lstset{numbers=left, numberstyle=\tiny, stepnumber=1, numbersep=-2pt}
\begin{lstlisting}[frame=single]
%UKF_UPDATE1 -  Additive form Unscented Kalman Filter update step
% Syntax:
%   [M,P,K,MU,S,LH] = UKF_UPDATE1(M,P,Y,h,R,param,alpha,beta,kappa,mat)
% In:
%   M  - Mean state estimate after prediction step
%   P  - State covariance after prediction step
%   Y  - Measurement vector.
%   h  - Measurement model function as a matrix H defining
%        linear function h(x) = H*x, inline function,
%        function handle or name of function in
%        form h(x,param)
%   R  - Measurement covariance.
%   h_param - Parameters of h (optional, default empty)
%   alpha - Transformation parameter (optional)
%   beta  - Transformation parameter (optional)
%   kappa - Transformation parameter (optional)
%   mat   - If 1 uses matrix form (optional, default 0)
% Out:
%   M  - Updated state mean
%   P  - Updated state covariance
%   K  - Computed Kalman gain
%   MU - Predictive mean of Y
%   S  - Predictive covariance Y
%   LH - Predictive probability (likelihood) of measurement.
% Description:
%   Perform additive form Discrete Unscented Kalman Filter (UKF)
%   measurement update step. Assumes additive measurement
%   noise.
%   Function h should be such that it can be given
%   DxN matrix of N sigma Dx1 points and it returns 
%   the corresponding measurements for each sigma
%   point. This function should also make sure that
%   the returned sigma points are compatible such that
%   there are no 2pi jumps in angles etc.
% Example:
%   h = inline('atan2(x(2,:)-s(2),x(1,:)-s(1))','x','s');
%   [M2,P2] = ukf_update(M1,P1,Y,h,R,S);
% See also:
%   UKF_PREDICT1, UKF_PREDICT2, UKF_UPDATE2, UKF_PREDICT3, UKF_UPDATE3,
%   UT_TRANSFORM, UT_WEIGHTS, UT_MWEIGHTS, UT_SIGMAS
function [M,P,K,MU,S,LH] = ukf_update1(M,P,Y,h,R,h_param,alpha,beta,kappa,mat)
  %
  % Check that all arguments are there
  if nargin < 5
    error('Too few arguments');
  end
  if nargin < 6
    h_param = [];
  end
  if nargin < 7
    alpha = [];
  end
  if nargin < 8
    beta = [];
  end
  if nargin < 9
    kappa = [];
  end
  if nargin < 10
    mat = [];
  end
  % Apply defaults
  %
  if isempty(mat)
    mat = 0;
  end
  % Do transform and make the update
  %
  tr_param = {alpha beta kappa mat};
  [MU,S,C] = ut_transform(M,P,h,h_param,tr_param);
  S = S + R;
  K = C / S;
  M = M + K * (Y - MU);
  P = P - K * S * K';
  if nargout > 5
    LH = gauss_pdf(Y,MU,S);
  end
\end{lstlisting}
\section{Filtro de Kalman de Cubatura}
\subsection{Ciclo de predicción}
\lstset{language=Matlab, breaklines=true, basicstyle=\footnotesize}
\lstset{numbers=left, numberstyle=\tiny, stepnumber=1, numbersep=-2pt}
\begin{lstlisting}[frame=single]
function [M,P] = ckf_predict(M,P,f,Q,f_param)
% CKF_PREDICT - Cubature Kalman filter prediction step
% Syntax:
%   [M,P] = CKF_PREDICT(M,P,[f,Q,f_param])
% In:
%   M - Nx1 mean state estimate of previous step
%   P - NxN state covariance of previous step
%   f - Dynamic model function as a matrix A defining
%       linear function f(x) = A*x, inline function,
%       function handle or name of function in
%       form f(x,param)(optional, default eye())
%   Q - Process noise of discrete model (optional, default zero)
%   f_param - Parameters of f (optional, default empty)
% Out:
%   M - Updated state mean
%   P - Updated state covariance
% Description:
%   Perform additive form spherical-radial cubature Kalman filter (CKF)
%   prediction step.
%
%   Function f(.) should be such that it can be given a
%   DxN matrix of N sigma Dx1 points and it returns 
%   the corresponding predictions for each sigma
%   point. 
  % Check which arguments are there
  if nargin < 2
    error('Too few arguments');
  end
  if nargin < 3
    f = [];
  end
  if nargin < 4
    Q = [];
  end
  % Apply defaults
  if isempty(f)
    f = eye(size(M,1));
  end
  if isempty(Q)
    Q = zeros(size(M,1));
  end
  % Do transform and add process noise
  if nargin < 5
    [M,P] = ckf_transform(M,P,f);      
  else
    [M,P] = ckf_transform(M,P,f,f_param);
  end
  P = P + Q;
\end{lstlisting}
\subsection{Ciclo de actualización}
\lstset{language=Matlab, breaklines=true, basicstyle=\footnotesize}
\lstset{numbers=left, numberstyle=\tiny, stepnumber=1, numbersep=-2pt}
\begin{lstlisting}[frame=single]

function [M,P,K,MU,S,LH] = ckf_update(M,P,Y,h,R,h_param)
% CKF_UPDATE - Cubature Kalman filter update step
% Syntax:
%   [M,P,K,MU,S,LH] = CKF_UPDATE(M,P,Y,h,R,param)
% In:
%   M  - Mean state estimate after prediction step
%   P  - State covariance after prediction step
%   Y  - Measurement vaector.
%   h  - Measurement model function as a matrix H defining
%        linear function h(x) = H*x, inline function,
%        function handle or name of function in
%        form h(x,param)
%   R  - Measurement covariance.
%   h_param - Parameters of h.
% Out:
%   M  - Updated state mean
%   P  - Updated state covariance
%   K  - Computed Kalman gain
%   MU - Predictive mean of Y
%   S  - Predictive covariance Y
%   LH - Predictive probability (likelihood) of measurement.
% Description:
%   Perform additive form spherical-radial cubature Kalman filter (CKF)
%   measurement update step. Assumes additive measurement noise.
%   Function h should be such that it can be given
%   DxN matrix of N sigma Dx1 points and it returns 
%   the corresponding measurements for each sigma
%   point. This function should also make sure that
%   the returned sigma points are compatible such that
%   there are no 2pi jumps in angles etc.
% Example:
%   h = inline('atan2(x(2,:)-s(2),x(1,:)-s(1))','x','s');
%   [M2,P2] = ckf_update(M1,P1,Y,h,R,S);

  % Check that all arguments are there
  if nargin < 5
    error('Too few arguments');
  end
  % Do transform and make the update
  if nargin < 6
    [MU,S,C,X] = ckf_transform(M,P,h);
  else  
    [MU,S,C,X] = ckf_transform(M,P,h,h_param);
  end
  S = S + R;
  K = C / S;
  M = M + K * (Y - MU);
  P = P - K * S * K';
  if nargout > 5
    LH = gauss_pdf(Y,MU,S);
  end
\end{lstlisting}